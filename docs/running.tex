
\newsection{ Running dynamics: \code{\MakeLowercase{hd}} }

Dynamics are computed using the program \code{hd}, short for Hamiltonian dynamics. 

\begin{bcomment}{Hamiltonian Dynamics}
Hamiltonian dynamics is a convenient formalism for propagating Newtonian dynamics with generalized variables (equivalent to constraints). Constraints and generalized variables (i.e., not necessary particle coordinate variables) are an essential feature of continuum membrane simulations, because the mesh is represented by control points and particles are frequently constrained to be on the membrane. The theory of Hamiltonian dynamics is discussed in~\cref{sec:hamilton}. 
\end{bcomment}

All arguments to \code{hd} are optional.
The first argument can be the name of an input file; this is the standard usage.
Subsequent arguments override simple input file directives:
\begin{command}
hd run.inp nsteps=10
\label{cmd:hd}
\end{command}


\newsubsection{Dynamics options and parameters }

The choice of timestep is critical for computing proper ensembles. 
If the timestep is too large, the system will be unstable and program execution will halt.
The option \code{timestep\_analysis=yes} provides a rough estimate of a proper timescale for dynamics.

If \code{timestep} is not specified, it defaults to
\begin{equation}
\textrm{\code{timestep}} = 10^{-3} \frac{65\textrm{ \AA}^2}{\pi \textrm{\code{diffc}}}
\end{equation}
where \code{diffc} is the diffusion constant in \AA$^2 s^{-1}$.

\subsubsection{Thermostatting: Langevin dynamics}

\begin{bcomment}{Langevin Dynamics}
Langevin dynamics is frequently selected as a model of implicit solvent.
The solvent degrees of freedom are not modeled.
In this framework, the effect of solvent is reduced to a random force that bumps those degrees of freedom that are modeled (for example, the membrane).
These random forces impart thermal energy into the system.
Viscosity of the solvent is modeled by a friction coefficient the reduces the momentum of the modeled degrees of freedom.
A balance between viscosity and the random force ensures the correct temperature and Boltzmann ensemble.
It can thus be chosen to model solvent or used as a simple thermostat to maintain temperature.
The mathematics of Langevin dynamics are discussed in~\cref{sec:langevin}
\end{bcomment}

Temperature can be controlled through the simultaneous application of a noise force and friction applied to the surface and particle momenta.
In the low friction limit, this corresponds to Newtonian mechanics.
In high friction limit, this corresponds to Brownian dynamics.
The friction coefficient is set using the \code{gamma\_langevin} parameter, which sets $\gamma$ to be this fraction of the timestep $\tau$.
That is,
\begin{equation}
\gamma = \frac{\textrm{\code{gamma\_langevin}}}{\tau} 
\end{equation} 

\subsubsection{Thermostatting: Brownian dynamics}

\begin{bcomment}{Brownian Dynamics}
Brownian dynamics is equivalent to Langevin dynamics in the high friction limit.
For a given timestep, solvent and frictional forces are sufficient to decorrelate velocities over the timestep.
As a result, a degree-of-freedom ``hops'' from one value to the next depending on noise and whatever potential energies are driving motion.
This model is consistent with a diffusion constant as an external parameter, rather than one that emerges from the underlying dynamics.  
The mathematics of Brownian dynamics are discussed in~\cref{sec:brownian}.
\end{bcomment}

An isothermal ensemble can also be determined with Brownian dynamics.
To activate Brownian dynamics for particles, use \code{do\_bd\_particles yes}.
To activate Brownian dynamics for the membrane, use \code{do\_bd\_membrane yes}.

\subsubsection{Membrane kinetics}

Well-defined kinetics can be activated when membrane modes are used by specifying \code{kinetics yes}.
Currently this requires the use of membrane modes, and is limited to slightly perturbed sphere and planar systems. 

\subsubsection{Using an area compressibility modulus, $K_A$}

\begin{bcomment}{The area compressibility modulus}
The energy (per area) of a membrane patch experiencing area strain $\epsilon$
\begin{equation}
\epsilon = \frac{A-A_0}{A_0},
\end{equation}
determined from its instantaneous area $A$ and resting area $A_0$, is
\begin{equation}
\bar{E}_A = \frac{K_A}{2} \epsilon^2.
\end{equation}
\end{bcomment}

The specification of the area compressibility modulus of the mesh is a key point.
Applying a target area to each mesh triangle prevents a catastrophe of the in-plane kinetic energy of the mesh control points.
Membrane curvature should be insensitive to this in-plane motion, yet it will destroy the ability of the mesh to represent complex objects.
Furthermore, these motions do not model true membrane degrees of freedom.
By specifying the area compressibility modulus, in-plane motion of the control points models thickness/area fluctuations of the membrane.

With normal modes, in-plane motion is completely supressed and thus there is no reason to use $K_A$.
In this case it should be set to zero (the default with modes activated).

\newsubsection{Loading a simulation state}

Dynamics and minimization can be restarted by loading a save file.
The input syntax is \code{load <name.save>}:
\begin{command}
hd run.inp load=min.save
\end{command}
This command can be put in the input file as well.
Save files are generated at the end of minimization and dynamics.
At the end of minimization, the file \code{min.save} is generated.
At the end of a dynamics simulation, the file \code{jobName.save} is created, where \code{jobName} is the overall job name given in the input file.
The default for this name is \code{default}, so the default for the save file is \code{default.save}.




