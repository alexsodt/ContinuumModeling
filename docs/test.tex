\documentclass[11pt]{article}
\usepackage{amsmath}
\usepackage{geometry}
\geometry{margin=0.75in}

\newcommand{\vek}[1]{\boldsymbol{#1}}          % vector symbol

\begin{document}

\subsection{Collective variable Brownian dynamics}

Given a set of $N$ momenta $\{p_i\}$, friction coefficient $\gamma$, and effective mass matrix $\hat{M_{ij}}$, there are $N$ equations:
\begin{equation}
\label{eq:diffc}
p_i'(t) = - \gamma \hat{M} \cdot \{p\} + f_i 
\end{equation}
A trial $a$ solution has
\begin{align}
\{ p \} &= \{ c_{ai}, c_{aj}, \ldots, .. c_{aN} \} \exp(-\lambda_a t) \\
        &= \vek{c}_a \exp(-\lambda_a t)
\end{align}
For two solutions $a$ and $b$, a linear combination of the two would also be a solution, and so they can be assembled to reproduce arbitrary initial conditions ($\{ p(0) \}$) at $t=0$.

The solution must fulfill Eq.~\ref{eq:diffc} as:
\begin{equation}
-\lambda_a \vek{c}_a \exp(-\lambda_a t) = -\gamma \hat{M} \cdot \vek{c}_a \exp(-\lambda_a t)
\end{equation}
That is:
\begin{equation}
\gamma \hat{M} \cdot \vek{c}_a = \lambda_a \vek{c}_a,
\end{equation}
which is only true for eigenvalue/eigenvector pairs $\{\lambda_a, \vek{c}_a\}$.

We now introduce constant external force $\{f_i\}$:
\begin{equation}
\label{eq:diffc2}
p_i'(t) = - \gamma \hat{M} \cdot \{p_j\} + f_i 
\end{equation}
and attempt the trial solution:
\begin{align}
\{ p \} &= \{ c_{ai}, c_{aj}, \ldots, .. c_{aN} \} \exp(-\lambda_a t) + \vek{k} \\
        &= \vek{c}_a \exp(-\lambda_a t) + \vek{k}
\end{align}
where $\vek{k}$ is of length $N$.
The solution is as before but now with
\begin{equation}
\gamma \hat{M} \cdot \vek{k} = \vek{f}
\end{equation}

\subsection{Molecular justification for $\gamma$}

The surface is formed of many molecular units $\{ r_i \}$ with mass $m_i$.
The microscopic Langevin equation for unit $i$ is:
\begin{equation}
\label{eq:microl}
m v_i'(t) = -\gamma v_i(t) + f, 
\end{equation}
with $f$ including a random force.
The factor $\gamma$ (units mass $\times$ time$^{-1}$) specifies the magnitude of the frictional force, typically interpreted as a collision rate.
However, the propagation of these units collectively is accomplished with collective variables $\{ q \}$, here the control points of the mesh.
The points $\{ r_i \}$ and thus their velocities are functions of the generalized coordinates:
\begin{equation}
v_i = \sum_j \frac{\partial r_i}{\partial q_j} \dot{q}_j
\end{equation}
Inserting this expression into Eq.~\ref{eq:microl} yields:
\begin{equation}
m_i \sum_j \frac{\partial r_i}{\partial q_j} \ddot{q}_j = -\gamma \sum_j \frac{\partial r_i}{\partial q_j} \dot{q}_j + f_i 
\end{equation}
where with the subdivision limit algorithm the surface position is proportional to the mesh control points and so second derivatives only enter for $\ddot{q}$.
With
\begin{equation}
\dot{q}_i = \sum_j M^{-1}_{ij} p_j 
\end{equation}
this transforms to
\begin{equation}
m_i \sum_j \frac{\partial r_i}{\partial q_j} \sum_{k} M^{-1}_{jk} \dot{p}_k = -\gamma \sum_j \frac{\partial r_i}{\partial q_j} \sum_k M^{-1}_{jk} p_k
\end{equation}
As there are many sites $\{ r_i \}$, this equation is an over-determined system of differential equations for the momenta $\{p_k(t)\}$.
\end{document}









