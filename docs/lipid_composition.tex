
\newsection{ Setting the bilayer lipid composition } \label{sec:lipid_comp}

The bilayer lipid composition is set in the input file; unlike other input commands it cannot be overridden with command-line options.
Lipid composition commands begin with \code{lipid} and then are followed by sub-commands.
For example, to set the inner leaflet to be DOPC, include
\begin{icommand}
lipid inner DOPC 100
\end{icommand}
in the input file.
The leaflet is selected with either \code{inner} or \code{outer}. 
The fourth argument is the amount of lipid, in parts.
To an outer leaflet with 50\% DOPC and 50\% DOPE, include, for example 
\begin{icommand}
lipid inner DOPC 100 \\
lipid inner DOPE 100
\end{icommand}
or
\begin{icommand}
lipid inner DOPC 50 \\
lipid inner DOPE 50
\end{icommand}
where with the parts mechanism the total amount of lipid always sums to 100\%.

To add a new lipid to the library, use the sub-command \code{library}:
\begin{icommand}
lipid library SAPC 80.0 -0.01
\end{icommand}
where \code{SAPC} is the name of the lipid, \code{80.0} is the area-per-lipid, and \code{-0.01} is the spontaneous curvature.
This lipid can now be used in a compositional command:
\begin{icommand}
lipid inner POPC 70 \\
lipid inner SAPC 30
\end{icommand}
The input file is parsed iteratively, beginning with \code{library} sub-commands, so the order in the input file is not significant.



%\newsection{Bibliography}
%\bibliography{hamiltonian}
%
%\end{document}

%document


