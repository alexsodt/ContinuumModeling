
\section{ Timescales } \label{sec:timescales}

\subsection{Langevin dynamics}

Newton's equation of motion is:
\begin{equation}
m \dot{\vek{v}} = \vek{f}
\end{equation}
where $\vek{v}$ and $\vek{f}$ are time dependent quantities.
With {\bf Langevin dynamics}, a frictional drag and stochastic force are introduced.
This is frequently justified as arising from implied collisions with solvent, itself in thermal equilibrium with a bath.
The modified equations are
\begin{equation}
m \dot{\vek{v}} = -\frac{\vek{v}}{\mu} + \vek{f} + \vek{\tilde{f}}
\end{equation}
or often with a friction coefficient $\gamma$ replacing the mobility $\mu$:
\begin{equation}
\label{eq:langevind}
m \dot{\vek{v}} = -\gamma \vek{v} + \vek{f} + \vek{\tilde{f}}
\end{equation}
where $\vek{\tilde{f}}$ is the stochastic force.

Consider modeling a particle diffusing with Langevin dynamics.
In the absence of any forces, $\vek{f}$, the diffusion constant is
\begin{equation}
D = \mu \kT
\end{equation}
To simulate a lipidic diffusion constant, e.g., $10^{-6} \textrm{cm}^2/\textrm{s}$, requires specifying $\mu$ appropriately.
Note that in Eq.~\ref{eq:langevind} the velocity is reduced by a fraction equal to $\frac{-\Delta t}{m \mu}$,
where $\Delta t$ is the simulation timestep.
This fraction must be much less than one or the particle will experience uncontrolled feedback and therefore improper integration.

\subsection{Brownian dynamics}

Under constant force, the velocity obeys
\begin{equation}
\dot{v} = -\gamma v + f/m
\end{equation}
The solution to this simple differential equation is
\begin{equation}
v(t) = v_0 \exp(-\gamma t) + \frac{f}{\gamma m}
\end{equation}
That is, the velocity decays with characteristic timescale $\tau=\frac{1}{\gamma}$.
If timesteps on the order of (or larger than) $\tau$ are desired, {\it it makes no sense to attempt to propagate the velocity}.
During the integration period the velocity will decorrelate completely from its initial value.
Rather, consider an approximate dynamics that is displacement-based, that is, that attempts to model a particle's motion under thermal agitation in the presence of external forces.
This is {\bf Brownian dynamics}. 
Here the equation of motion is 
\begin{equation}
\dot{\vek{r}} = \frac{D}{\kT} \vek{f} + \sqrt{2 D} R(t)
\end{equation}
where $R(t)$ is a so-called ``Gaussian process'', a mathematical object equivalent to selecting {\it uncorrelated} random numbers from a Gaussian distribution with zero mean and standard deviation one.

Because the momentum is dissipated in less than a time step in Brownian dynamics, it is not a relevant quantity for propagation.
For a basic particle simulation, this is handled simply; the particle has a diffusion constant $D$ and moves independently from the other particles, subject to its own noise, $R(t)$.
In a continuum membrane simulation, the particles of the membrane are interconnected through the generlalized coordinates: the mesh control points.
The Langevin equations can thus be transformed to be compatible with extracting the momentum.
From:
\begin{equation}
\dot{p}_i = \frac{\diff H}{\diff q_i} - \dot{p}_i \lambda_i + \eta_i(t) \nonumber \\
\dot{q}_i = \sum_j M_{ij}^{-1} p_j
\end{equation}
to
\begin{equation}
\dot{q}_i = -\gamma M_{ij}^{-1} \frac{\diff H}{\diff q_j} + \sum_{ij} M^{-1} \sqrt{2} 
\end{equation}










\subsection{Membrane relaxation timescales}




