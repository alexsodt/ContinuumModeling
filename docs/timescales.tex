
\newsection{ Timescales } \label{sec:timescales}

\newsubsection{Langevin dynamics}

Newton's equation of motion is:
\begin{equation}
m \dot{\vek{v}} = \vek{f}
\end{equation}
where $\vek{v}$ and $\vek{f}$ are time dependent quantities.
With the {\bf Langevin dynamics} of Eq.~\ref{eq:biglangevin}, a frictional drag and stochastic force, $\vek{\tilde{f}}$, are introduced.
This is frequently justified as arising from implied collisions with solvent, itself in thermal equilibrium with a bath.
The modified equations are
\begin{equation}
m \dot{\vek{v}} = -\frac{\vek{v}}{\mu} + \vek{f} + \vek{\tilde{f}}
\end{equation}
With a constant applied force, the velocity is $f \mu$.
The factor $mu$, the ratio of this velocity (termed the drift velocity) to an applied constant force, is called the mobility. 
It is the inverse of the friction coefficient $\gamma$ (equivalently, the momentum $\lambda$ in Eq.~\ref{eq:biglangevin}).
%or often with a friction coefficient $\gamma$ replacing the mobility $\mu$:
%\begin{equation}
%\label{eq:langevind}
%m \dot{\vek{v}} = -\gamma \vek{v} + \vek{f} + \vek{\tilde{f}}.
%\end{equation}
Consider modeling a particle diffusing with Langevin dynamics.
In the absence of any forces, $\vek{f}$, the diffusion constant is
\begin{equation}
D = \mu \kT
\end{equation}
%The Stokes-Einstein relation
%\begin{equation}
%D = \frac{\kT}{6 \pi \eta r}
%\end{equation}
%provides the diffusion constant in terms of the viscosity $\eta$ and the particle radius $r$.
To simulate a lipidic diffusion constant, e.g., $10^{-6} \textrm{cm}^2/\textrm{s}$, requires specifying $\mu$ appropriately.
Note that in Eq.~\ref{eq:langevind} the velocity is reduced by a fraction equal to $\frac{-\Delta t}{m \mu}$,
where $\Delta t$ is the simulation timestep.
This fraction must be much less than one or the particle will experience uncontrolled feedback and therefore improper integration.


\newsubsection{Brownian dynamics}

If the potential a particle experiences is smooth, it will be under an approximately constant force for the duration of a sufficiently short time step.
Under constant force, the velocity obeys
\begin{equation}
\dot{v} = -\frac{\gamma}{m} v + f/m
\end{equation}
The solution to this simple differential equation is
\begin{equation}
v(t) = v_0 \exp(-\frac{\gamma}{m} t) + \frac{f}{\gamma m}
\label{eq:vbrownian}
\end{equation}
That is, the velocity decays with characteristic timescale $\tau=\frac{m}{\gamma}$.
If timesteps on the order of (or larger than) $\tau$ are desired, the velocity obeys Eq.~\ref{eq:vbrownian} and this can be incorporated into the equations of motion.
During the integration period the velocity will decorrelate completely from its initial value.
Rather than propagating the velocity, consider an approximate dynamics that models the frictional slowing of the particles. 
With velocity removed, the equations are {\it displacement-based}, that is, they attempt to model a particle's motion under friction, with thermal agitation, in the presence of external forces.
This is {\bf Brownian dynamics}. 
Here the equation of motion is 
\begin{equation}
\dot{\vek{r}} = \frac{D}{\kT} \vek{f} + \sqrt{2 D} \eta(t)
\end{equation}
where $\eta(t)$ is a random process defined by
\begin{equation}
\langle \eta(t) \eta(t')\rangle = \delta(t-t')
\end{equation} 
that is produced {\it in silico} by a sequence of uncorrelated random numbers with zero mean and unit standard deviation.

Because the momentum is dissipated in less than a time step in Brownian dynamics, it is not a relevant quantity for propagation.
Instead, the random forces applied over the time step, perturbed by the constant force, lead to a shift in the particle coordinates that is uncorrelated with previous steps (with the exception of the influence of the force).
The result is diffusive dynamics.

For a basic particle simulation, this is handled simply; the particle has a diffusion constant $D$ and its coordinates are only coupled to other particles by the potential energy.
In a continuum membrane simulation, the particles of the membrane are interconnected through the generalized coordinates: the mesh control points.
The Langevin equations must be transformed to be compatible with eliminating the momentum.
From:
\begin{align}
\dot{p}_i &= -\frac{\partial H}{\partial q_i} -p_i \bar{\gamma}_i + \sqrt{2 \gamma_i \kT} \eta_i(t) \nonumber \\
\dot{q}_j &= \sum_j M_{ji}^{-1} p_i
\end{align}
to
\begin{equation}
\dot{q}_j = -M_{ji}^{-1} \bar{\gamma}_i^{-1} \frac{\partial H}{\partial q_i} + M_{ji}^{-1}\bar{\gamma_i}^{-1}\sqrt{2 \gamma_i \kT}\eta_i(t)
\label{eq:dotqmassed}
\end{equation}
where $\eta_i(t)$ are independent random processes, and $\bar{\gamma}$ is the per-mass coupling constant with units time$^{-1}$.
The surface mass matrix appears in Eq.~\ref{eq:dotqmassed}. 



\subsubsection{Brownian dynamics with collective variables}


A variable with a quadratic potential, under Brownian dynamics, has simple kinetics:
\begin{equation}
\dot{q} = -\gamma^{-1} k q + \sqrt{ 2 \gamma^{-1} \kT } \eta
\end{equation}
Ignoring the random force that has no systematic effect yields:
\begin{equation}
q(t) \approx q_0 \exp{-\gamma^{-1} k t}
\end{equation}

Consider a membrane undulation of a Helfrich bilayer.
The energy of a particular mode
\begin{equation}
h(x,y) = h_q \sin(x q_x + y q_y )
\end{equation}
is
\begin{equation}
E = \frac{1}{4} k_c h_q^2 A q^4
\end{equation}
The restoring force on $h_q$ is thus $\frac{1}{2} k_c h_q A q^4$.

To set the relaxation time of the mode to a particular value of $\tau$, for example
\begin{equation}
\tau_m = \frac{4 \eta}{k_c q^3},
\end{equation}
requires specifying $\gamma_i$ such that the expected relaxation time $m^{-1} \gamma^{-1} k$ matches $\tau^{-1}$.
This is only possible when the mass matrix $M_{ij}$ of the modes is diagonal, that is, when the modes are uncoupled and so can have independently set timescales.







\newsubsection{Membrane relaxation timescales}




