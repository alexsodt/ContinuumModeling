
\newsection{ Hamiltonian and Lagrangian dynamics } \label{sec:hamilton}

\newsubsection{Lagrangian mechanics}


The Lagrangian $L$ is defined as:
\begin{equation}
L = T - V
\end{equation}
with $V$ the potential energy.\footnote{This section is adapted from the English language Wikipedia, beginning around 2019.}
The kinetic energy
\begin{equation}
T = \frac{1}{2} \sum m_k \nu_k^2
\end{equation}
is defined in terms of the real mass variables ($\nu_k$ and $r_k$). 
It is computed for the generalized coordinates $q$ as:
\begin{equation}
\vek{\nu}_k = \sum_j{\frac{\partial \vek{r}_k}{\partial q_j} \dot{q}_j} + \frac{\partial \vek{r}_k}{\partial t}
\end{equation}
Here $\vek{\nu}_k$ is the velocity of particle $k$ and $m_k$ is its mass.
In the case of the membrane itself the generalized coordinate is the control point.
The mass variables are sections of the membrane corresponding to points on the surface.
The quantities $\frac{\partial \vek{r}_k}{\partial q_j}$ are determined by the spline coefficients of the face, and each point is a linear combination of the (for regular faces) twelve control points.
The quantity $\frac{\partial L}{\partial q_j}$ corresponds to the force, $\frac{\partial V}{\partial x}$.
The differential equation of motion is 
\begin{equation}
\frac{\textrm{d}}{\textrm{d}t} {\huge (} \frac{\partial L}{\partial \dot{q}_j} {\huge )} = \frac{\partial L}{\partial q_j}
\end{equation}
Applied to the control points, the equation of motion is
\begin{equation}
M_{ij} \ddot{q}_j = \frac{\partial V}{\partial q_i}
\end{equation}
where 
\begin{equation}
M_{ij} = \sum_k m_k \frac{ \partial { \vek{r}_k } }{\partial q_j} \cdot \frac{\partial{\vek{r}_k}}{\partial q_i}
\end{equation}
This matrix is inverted to put the equations in form suitable for integration:
\begin{equation}
\ddot{q}_j = M^{-1}_{ij} \frac{\partial V}{\partial q_i}
\end{equation}

\newsubsection{Hamiltonian dynamics}

In Hamiltonian dynamics a new variable is introduced, the momentum $p$:
\begin{equation}
p_i = \frac{\partial L}{\partial \dot{q}_i}
\end{equation}
This equation is then solved for $\dot{q}_i$ and substituted into the Hamiltonian $H$:
\begin{equation}
H = \sum \dot{q}_i p_i - L
\label{eq:legendre}
\end{equation}
For the control points $p$ is to related to the set $q$ through:
\begin{align}
\label{eq:pqdot}
p_i &= \sum_j{ M_{ij} \dot{q}_j } \\
\label{eq:qdotp}
\dot{q}_j &= \sum_i{ M^{-1}_{ij} p_i}
\end{align}
The equations of motion are then:
\begin{align}
\dot{p} &= - \frac{\partial H }{\partial q} \\
\dot{q} &= + \frac{\partial H }{\partial p}
\end{align} 

\newsubsection{Particle on the membrane surface}

The coordinates of a particle on the membrane's surface are determined by their internal coordinates $u$ and $v$ as well as the control points of the surrounding mesh:
\begin{equation}
\vek{r}_i = \sum_{l} \vek{q}_l s_l u^{a_l} v^{b_l}
\end{equation}
where here the generalized variable $q$ is shown in vector form.
The powers $a_l$ and $b_l$ are non-negative integers less than five, and $s_l$ is the spline coefficient.
To calculate $T$ the velocity $\vek{\nu}$ must be expressed in terms of the generalized coordinates and their time derivatives:
\begin{equation}
\vek{\nu} =  \sum_l \frac{\partial \vek{r}}{\partial q_l} \dot{q}_l + \frac{\partial \vek{r}}{\partial u} \dot{u} +
\frac{\partial \vek{r}}{\partial v}\dot{v}
\end{equation} 
%Unlike the pure mesh model the particle coordinates couple multiple generalized coordinates ($q$, $u$, and $v$).

The particle's attachment modifies the dynamics of the membrane as its mass adds to that of the membrane.
The control point momenta are now determined as:
\begin{equation}
p_i = M_{ij} \dot{q}_j + m_p \frac{\partial \vek{r}}{\partial q_i} \cdot \frac{\partial \vek{r}}{\partial q_k} \dot{q}_k 
  + m_p \frac{\partial \vek{r}}{\partial q_i} \cdot \frac{\partial \vek{r}}{\partial u} \dot{u} 
  + m_p \frac{\partial \vek{r}}{\partial q_i} \cdot \frac{\partial \vek{r}}{\partial v} \dot{v} 
\label{eq:pmesh}
\end{equation}
%\begin{align}
%\vek{p}_i &= M_{ij} \dot{\vek{q}}_j + m_p \sum_{ik} (s_i u^{a_i} v^{b_i}) (s_k u^{a_k} v^{b_k}) \dot{\vek{q}}_k + \nonumber \\
%    & + m_p \sum_{ik}  s_i u^{a_i} u^{b_i} \vek{q}_k s_k a_k u^{a_k-1} v^{b_k} \dot{u} + \nonumber \\ 
%    & + m_p \sum_{ik}  s_i u^{a_i} u^{b_i} \vek{q}_k s_k b_k u^{a_k} v^{b_k-1} \dot{v}  
%\label{eq:pmesh}
%\end{align}
where the sums for the attached particle are only over control points that are included in the spline.

The attached particle's velocity depends on $\dot{q}_i$, $\dot{u}$ and $\dot{v}$, so all cross-terms in $T$ must be accounted for in the calculation of, for example, $\frac{\partial T}{\partial \dot{u}}$.
\begin{equation}
p_{u} = m_p [\frac{\partial \vek{r}}{\partial q_k} \cdot \vek{r}_u \dot{q}_k +  \vek{r}_u \cdot \vek{r}_u \dot{u}  +  \vek{r}_u \cdot \vek{r}_v \dot{v} ] 
\label{eq:pu}
\end{equation}
and
\begin{equation}
p_{v} = m_[ [\frac{\partial \vek{r}}{\partial q_k} \cdot \vek{r}_v \dot{q}_k +  \vek{r}_v \cdot \vek{r}_u \dot{u}  +  \vek{r}_v \cdot \vek{r}_v \dot{v} ] 
\label{eq:pv}
\end{equation}
The routines to evaluate these terms are \code{surface::ru} ($\vek{r}_u$), \code{surface::rv} ($\vek{r}_v$), and \code{surface::get\_pt\_coeffs} ($\frac{\partial \vek{r}}{\partial q_k}$), available in \code{uv\_map.C}.

In theory, the $\dot{q}_i$, $\dot{u}$, and $\dot{v}$ are then substituted into Eq.~\ref{eq:legendre} to determine the new $H$.
Setting aside briefly the momentum of the underlying mesh, $\dot{\vek{q}}$, the surface coordinates of the particle momentum can be cast as:
\begin{align}
\begin{bmatrix}
p_{u} \\
p_{v} 
\end{bmatrix}
= m_p \sum_{kl}{
\begin{bmatrix}
\vek{r}_u \cdot \vek{r}_u  & \vek{r}_u \cdot \vek{r}_v \\
\vek{r}_v \cdot \vek{r}_u  & \vek{r}_v \cdot \vek{r}_v 
\end{bmatrix}}
\cdot
\begin{bmatrix}
\dot{u} \\
\dot{v} 
\end{bmatrix}
\label{eq:pmat}
\end{align} 
or
\begin{equation}
p_\alpha = m_p g_{\alpha\beta} \dot{\beta}
\end{equation}
where $\alpha$, $\beta$ are $u$ or $v$.
Conceptually, the factors act to slow down $\dot{u}$ as it approaches regions with high metrics.
It does not do this with a force, rather, $\dot{u}$ is computed instantaneously from $p$, which is the variable being propagated.

Unlike $g_{uv}$, the control point mass matrix $M_{ij}$ is determined only by network topology, if the masses represented by the mesh are not chosen to change. 
Thus, in the absence of bound particles, its inverse does not need to be calculated at each dynamics step to evaluate $\dot{\vek{q}}$ from $\vek{p}$ using Eq.~\ref{eq:qdotp}.
Eqs.~\ref{eq:pmesh},~\ref{eq:pu}, and~\ref{eq:pv} introduce state-dependent ($\vek{q}$, $u$, and $v$) coupling between $\dot{\vek{q}}$ and both the mesh and embedded particles.

To solve for the complete set of coordinate time derivatives, including both control point momenta and particle momenta, with the eventual goal of their elimination requires solving:
\begin{align}
\begin{bmatrix}
\vek{p}_i \\
\vek{p}_j \\
\vdots    \\
p_u \\
p_v
\end{bmatrix}
&=
\Bigg(
\begin{bmatrix}
M_{ii} & M_{ij} & \cdots & 0 & 0\\
M_{ji} & M_{jj} & \cdots & 0 & 0\\
\vdots & \vdots & \ddots & \vdots \\
0 & 0 & & m_p g_{uu} & m_p g_{uv} \\
0 & 0 & & m_p g_{vu} & m_p g_{vv}
\end{bmatrix}
\nonumber \\
&+ m_p 
\begin{bmatrix}
0 & \cdots & \frac{\partial \vek{r}}{\partial q_i} \cdot \vek{r}_u & \frac{\partial \vek{r}}{\partial q_i} \cdot \vek{r}_v \\
\vdots & \ddots & \vdots & \vdots \\
\vdots &        & \vdots & \vdots \\
\frac{\partial \vek{r}}{\partial q_i} \cdot \vek{r}_u  & \cdots & 0 & 0 \\
\frac{\partial \vek{r}}{\partial q_i} \cdot \vek{r}_v  & \cdots & 0 & 0 
\end{bmatrix}
\Bigg)
\cdot 
\begin{bmatrix}
\dot{\vek{q}_i} \\
\dot{\vek{q}_j} \\
\vdots    \\
\dot{u} \\
\dot{v}
\end{bmatrix}.
\end{align}
%where $P_{uv}(\{\vek{q}\},u,v)$ is the particle-particle coupling matrix defined in Eq.~\ref{eq:pmat} and $X_{iu}(\{\vek{q}\},u,v)$ is the mesh-particle coupling element defined in Eqs.~\ref{eq:pmesh},~\ref{eq:pu}, and~\ref{eq:pv}. 
The inverse of a slightly perturbed matrix $A_0 + \delta A$ can be computed with the truncated series
\begin{equation}
(A_0+\delta A)^{-1} = A_0^{-1} - A_0^{-1} \delta A A_0^{-1} + \mathcal{O}[\delta A^2]
\end{equation}
because
\begin{equation}
(A_0^{-1} - A_0^{-1} \delta A A_0^{-1}) \cdot (A_0 + \delta A) = I - A_0^{-1} \delta A A_0^{-1} \delta A
\end{equation}

The kinetic energy $T$ now has position-dependent cross-terms between the particle and membrane that contribute to the time derivative of the momenta.

\newsubsection{Propagating particle momenta $p$ }

Particle momenta $p$ are propagated using the force, e.g., $\frac{\partial{H}}{\partial{u}}$.
The kinetic energy now depends on $u$ through $\frac{\partial T}{\partial u}$ and $\frac{\partial T}{\partial v}$.
%For example:
%\begin{equation}
%\frac{\partial \sum{ \{p\} \cdot \{ \dot{q} \} }}{\partial u} =  \frac{\partial}{\partial u} \{ p \}^{T} \cdot (M^{-1}_{ub} + C_{ub}) \cdot \{ p \} 
%\end{equation}
%where the superscript $T$ indicates transposition.

\newsubsection{The effective mass matrix is banded; its inverse is short-ranged }

\subsection{Normal modes}

The mesh control points are ``generalized'' coordinates in the sense that they are not point centers of mass that directly feel force.
Rather, they determine the mass positions through the subdivision-limit algorithm.
Compared with the complexity of having each underlying lipid or its atoms independent entities, the use of a continuum mesh is a dramatic simplification.
It is often convenient to add a further layer of generalization by using so-called normal modes that are linear transformations of the mesh:
\begin{equation}
\vek{Q}_i = N_{ij} \vek{q}_j,  
\end{equation}
where $\vek{Q}$ is a normal mode and $\vek{q}$ is a mesh control point. 
Most frequently these modes are energetically decoupled at the level of the membrane elastic energy.
Creation of the normal modes requires solving for $N_{ij}$, again considering the membrane elastic energy.

% implement cylinders too?
The use of normal modes for spheres and planes is invoked by specifying either 
the details of a single mode (\code{mode\_x} and \code{mode\_y}) or the lower and upper limits of modes to model (\code{mode\_min} and \code{mode\_max}).  
For a spherical mesh, it is necessary to include the input \code{sphere yes} to invoke spherical harmonics. 



\newsubsection{Langevin dynamics}

The generic Langevin equation\footnote{The form of this equation was adapted from the English language Wikipedia page for the Langevin Equation on March 27th, 2019.}
is
\begin{equation}
\dot{A}_i = \sum_j \{ A_i, A_j \} \frac{\diff H}{\diff A_j} - \sum_j \lambda_{ij} \frac{\diff H}{\diff A_j} + \sum_j \frac{ \diff \lambda_{ij} }{\diff A_j} + \eta_i(t)
\label{eq:biglangevin}
\end{equation}
where here $\{ \cdot, \cdot \}$ is the Poisson bracket, $\lambda_i$ is the friction coefficient for (here, momentum) degree-of-freedom $A_i$, and $\eta(t)$ is a random noise process with
\begin{equation}
\langle \eta_i(t) \eta_i(t') \rangle = 2 \lambda_i \kT \delta(t-t')
\end{equation}
and zero mean.
The units of the damping coefficients $\lambda_i$ are $\frac{\textrm{mass}}{\textrm{time}}$ for coupling between momentum coordinates.
Note: when applying a frictional drag to the momentum, the $\lambda$ value is denoted $\gamma$ below.
For a simple Hamiltonian like $\frac{p^2}{2 m}$ the friction can be applied directly to the momentum trivially, i.e.:
\begin{align}
\dot{p} &= \frac{\diff H}{\diff q} - \lambda_p \frac{\diff H}{\diff p} + \eta_p(t) \\
\dot{p} &= - \lambda_p \frac{p}{m} + \eta_p(t) 
\end{align}
However, for a more complicated kinetic energy we use the relation between $p$ and $\dot{q}$, see Eq.~\ref{eq:qdotp}, with $\dot{q}$, itself computed from $p$, taking the place of $\frac{p}{m}$. 

% normally computed as nv by nv matrix.

%\newsection{Bibliography}
%\bibliography{hamiltonian}
%
%\end{document}

%document


