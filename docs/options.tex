\newenvironment{optionSummary}[3]{
    \mdfsetup{skipabove=0pt,skipbelow=0pt,frametitleaboveskip=0pt,frametitlebelowskip=5pt,innerbottommargin=0pt,hidealllines=true,frametitlefont=\bfseries\color{darkgray}}
    \begin{mdframed}
	[frametitle=\begin{flushleft}Option:~\MakeLowercase{\code{#1}}\\Default:~\code{#2}\end{flushleft},
	backgroundcolor=white]
     \addtostream{stream#3}{#1 \par}
  }{\par\begin{flushleft}\hrulefill\end{flushleft}\end{mdframed}}

\newsection{Input options}

\newcommand{\optionHeader}[1]{{\bf \begin{center}#1\end{center}}}

\newcommand{\optionRuler}{\vspace{-0.2in}\hrulefill}

\optionHeader{System setup options}
\optionRuler
\begin{multicols}{3}
\IfFileExists{setup.options}{\input{setup.options}}{}
\end{multicols}

\optionHeader{Dynamics options}
\optionRuler
\begin{multicols}{3}
\IfFileExists{dynamics.options}{
\newsection{ Membrane and particle dynamics } \label{sec:dynamics}

\newsubsection{Local metric}

The local metric of the membrane relates changes in a particle's surface coordinates $u$ and $v$ to changes in its three dimensional Cartesian vector. 

\newsubsection{Irregular mesh points}

An irregular mesh point has valence not equal to six.
At an irregular mesh point the metric and some related properties (curvature) diverge.
Note that the tangent plane does not.
Nevertheless this is a challenge for dynamics because critical quantities ($\dot{u}$, $\dot{v}$, $c(u,v)$) are changing rapidly with time.
The solution currently adopted is to split timesteps as necessary, on a particle-by-particle basis, as they approach irregular vertices.
Not all quantities are recomputed during the time splitting, only those that do not require interaction with other particles. 
}{}
\end{multicols}

\optionHeader{Reaction-diffusion options}
\optionRuler
\begin{multicols}{3}
\IfFileExists{rxn.options}{\input{rxn.options}}{}
\end{multicols}

\optionHeader{Miscellaneous options}
\optionRuler
\begin{multicols}{3}
\IfFileExists{misc.options}{\input{misc.options}}{}
\end{multicols}

\newoutputstream{streamSetup}
\openoutputfile{setup.options}{streamSetup}
\newoutputstream{streamDynamics}
\openoutputfile{dynamics.options}{streamDynamics}
\newoutputstream{streamRxn}
\openoutputfile{rxn.options}{streamRxn}
\newoutputstream{streamMisc}
\openoutputfile{misc.options}{streamMisc}

% BEGIN options

\newsubsection{System construction options}

\begin{optionSummary}{mesh}{planar.mesh}{Setup}
Specifies the file name of the mesh to use for the simulation.
\end{optionSummary}

\begin{optionSummary}{add}{N/A}{Setup}
Adds particles/complexes to the membrane.

Syntax:
\code{add <complex\_name> nbound <nbound> <inside/outside>}
\end{optionSummary}

\newsubsection{Dynamics options}

\begin{optionSummary}{do\_ld}{off}{Dynamics}
Activates the Langevin dynamical thermostat to propagate both surfaces and particles.
\end{optionSummary}
\begin{optionSummary}{gamma\_langevin}{10 AKMA time}{Dynamics}
Sets the value of $\gamma$, the coupling constant that controls the rate of collisions with the virtual solvent. The units are in AKMA time.
\end{optionSummary}
\begin{optionSummary}{do\_bd}{off}{Dynamics}
Activates Brownian dynamics to propagate membrane and particles.
\end{optionSummary}
\begin{optionSummary}{do\_bd\_particles}{off}{Dynamics}
Activates Brownian dynamics only for particles (rather than the membrane).
\end{optionSummary}
\begin{optionSummary}{time\_step}{one nanosecond}{Dynamics}
Time step used to propagate dynamics.
Too large a time step in the system will lead to positive feedback of high forces and large motions, crashing the system.
Appropriate timesteps will conserve energy (when not employing a thermostat).
Too small timesteps will waste computational resources.
The default will rarely be appropriate.
\end{optionSummary}
\begin{optionSummary}{nouter}{10000}{Dynamics}
Number of ``outer'' steps of dynamics.
Trajectory information is written every \code{nouter} steps.
The total number of time steps is nouter $\times$ ninner.
\end{optionSummary}
\begin{optionSummary}{ninner}{10000}{Dynamics}
Number of ``inner'' steps of dynamics.
Each outer loop of dynamics loops over this \code{ninner} time steps.
The total number of time steps is nouter $\times$ ninner.
\end{optionSummary}

\newsubsection{Reaction diffusion options}

\begin{optionSummary}{do\_rd}{off}{Rxn}
Activates reaction/diffusion methodology.
\end{optionSummary}
\begin{optionSummary}{rxn\_diffusion}{none}{Rxn}
Input file for reaction diffusion.
Currently the format is:\par
\code{reactant\_name1 site\_type1 reactant\_name2 site\_type2 k\_on(vol/s) k\_off(/s) binding\_radius(Angs) productName(or generic instructions)}
\end{optionSummary}

\newsubsection{Miscellaneous options}

\begin{optionSummary}{disable\_mesh}{off}{Misc}
Disables all propagation of mesh coordinates.
\end{optionSummary}

% these must be at the end of the file
\closeoutputstream{streamSetup}
\closeoutputstream{streamDynamics}
\closeoutputstream{streamRxn}
\closeoutputstream{streamMisc}
